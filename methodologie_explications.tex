\PassOptionsToPackage{gray,blue}{xcolor} % pour l'option de xcolor
\documentclass{beamer} 
\usepackage[T1]{fontenc}
\usepackage[utf8]{inputenc} 
\usepackage{beamerthemeshadow}
\usecolortheme{seagull} % couleur du thème Warsaw beaver
\setbeamercolor*{title}{bg=lightgray,fg=black}
\setbeamercolor{caption name}{fg=blue}
\setbeamertemplate{footline}{}
\usepackage{amsmath} 
\newcommand{\N}{\mathbb{N}}
\usepackage{amssymb}
\usepackage{sidecap} % caption on the side
\usepackage{bm} % maths en gras
\usepackage{colortbl} % Couleurs dans les tableaux
\usepackage{multicol} 
%\usepackage{xcolor}
\usepackage{caption}
\captionsetup[figure]{labelfont={color=blue}}
\usepackage{upquote} % pour les <<'>> dans le texte
%\usepackage{pgfpages} % pour imprimer sur deux pages
%\mode<handout>{\setbeamercolor{backgroundcanvas}{bg=black!20}} % suite
%\pgfpagesuselayout{2 on 1}[a4paper,border shrink=5mm] % suite

%\usepackage[gray]{xcolor}
\setbeamersize{text margin left=0.8em} \setbeamersize{text margin right=1em}

\usepackage{listings} % Ecriture du code
\newcommand{\monespacegris}{\escapechar={\gray{\textvisiblespace}}}
\definecolor{fond}{gray}{0.90} % Couleur de fond des lignes de code
\definecolor{commentscol}{gray}{0.4} % Couleur de fond des lignes de commentaire de code
\definecolor{tabcol}{gray}{0.75} % Couleur de fond des indentations
\lstset{tab=\color{tabcol}\rightarrowfill} % tabulations sous formes de flèches
\lstset{language=Python} % language
\lstset{backgroundcolor=\color{fond}} % couleur de fond
\lstset{commentstyle=\ttfamily\color{commentscol}\slshape} % style des commentaires
\lstset{identifierstyle=\ttfamily\mdseries} % style des identifiants (?)

\lstset{stringstyle=\ttfamily\mdseries} % fonte à chasse fixe pour le code
\lstset{showstringspaces=false}
\lstset{showtabs=true} % tabulations visibles
\lstset{tabsize=4} % longueur des tabulations
\lstset{xleftmargin=5pt} % décallage vers la droite de tout le code
\lstset{mathescape=true} %
\lstset{escapebegin=\color{commentscol}} 
\lstset{frame = single}
\lstset{rulecolor = \color{black}} % couleur de l'entourage
% \lstset{aboveskip = \smallskipamount} % au lieu de \medskipamount
\lstset{framerule = 1pt} 
\lstset{inputencoding=utf8, extendedchars=true }
\lstset{literate={à}{{\`a}}1{é}{{\'e}}1{è}{{\`e}}1{ç}{{\c
      c}}1{ù}{{\`u}}1{ê}{{\^e}}1{î}{{\^\i}}1{ï}{{\"\i}}1{ô}{{\^o}}1{ë}{{\"e}}1{$'$}{{'}}1{â}{{\^a}}1{û}{{\^u}}1}
% \lstset{upquote=true} % pour les ' ' droits
\lstset{inputencoding=utf8} % ne semble pas résoudre les problèmes d'encodage 
\lstset{morekeywords={plt.plot}} 
\lstset{morekeywords={show}} 
\lstset{morekeywords={as}}
\lstset{morekeywords={lambda}}
\lstset{morekeywords={as}}
\lstset{morekeywords={def}} 
\lstset{morekeywords={append}}
\lstset{morekeywords={for}} 
\lstset{morekeywords={import}}
\lstset{morekeywords={cos}}
\lstset{morekeywords={lambda}}
\lstset{morekeywords={range}} 
\lstset{morekeywords={in}}
\lstset{morekeywords={grid}} 
\lstset{morekeywords={np}} 
\lstset{morekeywords={legend}} 
\lstset{morekeywords={plot}} 
\lstset{morekeywords={arange}}
\lstset{morekeywords={linspace}}
\lstset{morekeywords={plt}}
\lstset{keywordstyle=\bfseries\ttfamily} % style des mots clefs
% \lstset{frame=single} % entourage du code
% \usepackage{graphicx}
\usepackage[frenchb]{babel} \setlength{\arrayrulewidth}{1.2pt}



\title[]{Procédure de sélection des futurs élèves de CPES} % Titre de la première page et en bas à droite
%\subtitle{Réunion de Fichier test}
\author{Commission de sélection}
\date{{mercredi 18 mai 2016}}
% _____________________________________________________________

\begin{document}
\section{\hfill Procédure de sélection - CPES} % Affiché en haut à gauche 
\frame{\titlepage} % Affichage du titre précédent

\logo{\includegraphics[height=1cm,scale=0.3]{naval2.jpg}} 



\section{\'Etude des dossiers APB}
\subsection{\'Etude des dossiers APB : qui ?}
\begin{frame}
Trois équipes sont constituées :
  \begin{itemize}
  \item Anglais-Lettres :  \qquad\;\;\, M\up{me} RIVIERE \& M. BARUCHEL 
  \item Mathématiques : \qquad \;\,\, M\up{me} DELORME  \& M. BARRAULT  
  \item Physique :  \qquad \quad \qquad\; M\up{mes} MOUCHE \& SAVIGNY
  \end{itemize}
\end{frame}
\subsection{\'Etude des dossiers APB : comment ?}
\begin{frame}
\begin{itemize}
\item Chaque enseignant attribue une note sur 10 par dossier ; 
\item Calcul de la moyenne des deux notes ($\rightarrow$ une note par discipline);
\item Relecture du dossier si écart trop fort des deux notes au sein d'une discipline (requêtes dans la base de données).
\end{itemize}
\end{frame}

\subsection{\'Etude des dossiers APB : complément}
\begin{frame}
Tout est pris en compte dans l'étude d'un dossier :
\begin{itemize}
\item Les commentaires a.p.b. des enseignants, du proviseur ;
\item Les notes (moyenne), le niveau de la classe, l'origine du dossier ;
\item L'option et la spécialité de terminale ;
\item Estimation de la capacité de l'élève à réagir positivement ;
\item Comportement général, ouverture d'esprit ;
\item Tout indice supplémentaire (familial, sportif, âge, degré de motivation, expériences, \ldots).
\end{itemize}
\end{frame}



\section{Classement des dossiers}
\subsection{Calcul de la note attribuée à chaque dossier}
\begin{frame}
Chaque dossier reçoit une note $N$ calculée ainsi :
\begin{large}
  \begin{equation*}
    N = \dfrac{\textit{Moy.}_{\textsf{Ang.-Let.}}+\textit{Moy.}_{\textsf{Math.-Sci.}}+\textit{Moy.}_{\textsf{Phys.-Sci.}}}{3}
  \end{equation*}
\end{large}
\end{frame}
\subsection{Classement avant commission}
\begin{frame}
Les dossiers sont classés (premier classement) :
  \begin{itemize}
  \item Par ordre décroissant de niveau de bourse (6, 5, 4, \ldots,0-bis, 0, non boursier(-1));
  \item Par ordre décroissant de note au sein de chaque degré de bourse.
  \end{itemize}
\end{frame}

\begin{frame}
Ont été ré-étudiés  tous les dossiers  : 
  \begin{itemize}
  \item  pour lesquels les deux notes au sein d'une même discipline présentent un écart important ;
  \item  pour lesquels l'écart entre deux des trois disciplines (math., phys., lettres) présentent un écart trop important ;
  \item  présentant une originalité (âge du candidat, candidats hors série S, bacheliers, candidat éloigné, \ldots).
  \end{itemize}
\end{frame}

\begin{frame}
Le classement obtenu (format papier) est issu :
  \begin{itemize}
  \item  du classement par ordre décroissant de bourse puis de note au sein de chaque degré de bourse ;
  \item  du déclassement de candidats pour diverses rasons liées à leur dossier (comportement, anomalies, \ldots) ;
  \item  du déclassement de ceux dont la note est inférieure à 4,8 ;
  \item  présentant une originalité (âge du candidat, candidats hors série S, bacheliers, candidat éloigné, \ldots).
  \end{itemize}
\end{frame}

\subsection{Classement effectué lors de la  commission}
\begin{frame}
La commission va permettre de 
  \begin{itemize}
  \item Valider/modifier l'ordre des priorités ;
  \item Valider/modifier la barre de 4,8 ;
  \item Apporter des compléments d'information.
  \item Autres \ldots
  \end{itemize}
\end{frame}

\begin{frame}
Comparaison avec l'année 2015:

\begin{center}
  \begin{tabular}[h]{|l|c|c|}
    \hline
    &  2015 & 2016 \\ \hline\hline
    dossiers complets   & 122    & 167\\ \hline 
    boursiers retenus   & 19     & 43 \\ \hline
    barre d'élimination & 3,99 & 4,8 \\ \hline
    nombre de candidats retenus & 84  & 90$^{(*)}$  \\ \hline
    classement du dernier admis & 61  & ? \\ \hline
  \end{tabular}
\end{center}
\begin{small}
  (*) 98 candidats (dont 5 nouveaux boursiers) si barre à 4,5.
\end{small}
\end{frame}



\end{document}


